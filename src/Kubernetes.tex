\chapter{Kubernetes}
Kubernetes is an open-source project aimed at solving issues related to networking and persistence
on different Docker hosts. Microservices in the scope of Kubernetes are redefined as one or more
Docker images inside a single pod, pods have the following advantages:
\begin{itemize}
	\item This preserves the benefits of containers for DevOps
	\item Pods share namespaces such as PID networks etc.
	\item Pods could also share persistent volumes
	\item The containers in the Pod are in localhost
\end{itemize}
A Kubernetes cluster consists of: a master node that coordinates, the cluster itself, the worker
nodes. A service is always discoverable internally to the cluster thanks to Kubernetes's DNS
service, the IP can also be discovered from outside if a valid external IP is assigned to it.

We can trace a parallel between Kubernetes and any Operating system then the Pods are processes,
Docker is the new apt and containers are new VMs.

If we compare the Kubernetes solution vs the Docker swarm solution we have to keep in mind that:
Kubernetes is widely used commercially, can operate autoscaling based on the amount of resources
used and has a very widespread community (it's open source); Docker swarm is very easy to use and is
included in the Docker Engine suite but it doesn't have a great community to back it up and it does
not have the same commercial success as Kubernetes.
