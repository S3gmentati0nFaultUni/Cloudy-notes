\chapter{Microservices containerization}
Microservices are a very interesting concept that came to be since 2012 and are essentially vertical
slices of the architecture, not layers, this means that there is no dependency on state, just
communication, furthermore each microservice is allowed to scale independently and employ the
best technological stack to solve the specific problem.

By definition of microservice there are a couple of properties that we can guarantee:
\begin{itemize}
	\item Replicability of environments
	\item Protection by isolation
	\item Distribution and coexistence
\end{itemize}
Containers encapsulate a runtime environment, because of how light the models are, they can be
initialized and sent into execution at very high speeds therefore achieving a quick response to any
possible situation. Microservices can be deployed using hardware, virtual machines or containers,
it depends on the structure of the solution and the non-functional requirements of the project.

\section{Docker}
Docker is a containerization software that encloses a piece of software in a complete filesystem
containing everything that is needed to run the software itself. It consists of a registry
containing images of containers, a daemon which queries the registry to get the image to run in a
container and the client that communicates with the deamon to take actions.

Docker is based on containers, which contains the running image and anything that is needed to run
it, has its own IP and exposes: a port to communicate with the outside world, volumes to save your
data, an ENV section to save and access environmental variables if needed.

Docker has access to a complete toolchain that allows us to manage containers not only via
dockerfiles but also via YAMLs, as well as manage clusters. To manage clusters Docker swarm it's
used, which deploys managers and workers and it's controlled through YAML. Managers are central nodes receiving service definitions and distributing work to workers.

Microservices are indispensable for an event driven IT model, containers are the perfect
virtualization technique to bring the microservice paradigm to life.
