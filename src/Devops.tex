\chapter{DevOps}
DevOps is a new form of software development that is founded on a series of tools and techniques
that improve the organization's ability to develop and deploy and service an application quickly and
efficiently. This new way of developing and supporting software was originated from the sprout of
programs like Docker that allow to make the CI/CD process a breeze. The DevOps process is founded on
the following steps:
\begin{itemize}
	\item PLAN considering the possibility to do iterations.
	\item DEVELOP code according to plan.
	\item BUILD an executable to be verified as soon as new code is available.
	\item TEST the executables.
	\item RELEASE the software.
	\item DEPLOY as a preparation for the production phase.
	\item verify IN OPERATION environment.
	\item MONITOR the software and plan improvements.
\end{itemize}
This solution is not always the best one, there is a certain point where the virtualization overhead
is too expensive for the type of application that we are trying to handle, therefore deploying using
a monolithic approach is a better solution afterall. There is also the derivation of DevOps which is DevSecOps.

The twelve factor is a methodology that pushes for declarative formats for automation and clear
contracts with the underlying operating system, the result is a focus on cloud deployment and a
greater ease of use. Docker is the perfect tool for this kind of pourpose.

The following is the A, B, C of twelve factor application development.

A codebase that is connected to a repository and receives constant updates from developers and also
has access to revision control.

Dependencies need to be declared and isolated, languages that are seeing more and more use in the
last couple of years like Go and Rust have really great mechanisms to handle dependencies.

Configuration files need to be present to allow handling of sensitive data without the risk of
having data in the clear, the idea is that if we have a config file and we move the project to open
source then we do not risk of giving our sensitive data to the users and the users utilizing our
software will not risk a security breach.

External services need to be interchangeable, there must be no need to change the codebase if a
certain provider has to change for whatever reason.

Divide the Building and Running steps so that after the building phase the code is ready to be run
and is basically an empty shell waiting for the final piece of information (the config file) during
the Running stage the code is then integrated with the config file to make the application
executable.

Everything should be stateless!!!

Always do port-binding for your own services, the result is that our own services can become backing
services for someone else.

Treat logs as streams.

Seeing how important has the cloud become we now have cloud-based applications, which means that
they are applications designed for the cloud and they are run and hosted on the cloud to capitalize
the cloud peculiarities, cloud-based applications are built using microservice architecture which
makes it easy to spin them up, shut them down, scale them and it also makes them resilient and
easily manageable.


