\section{Introduction}
First, we want to focus on the concept of a distributed system, which is, intuitively, a complex mechanism with many moving parts. The parts can move on their own or according to the plans of a master that is keeping everything under control. We will see a couple of different definitions that try to encapsulate what distributed systems are all about. \\
A distributed system can be seen as two or more computers, each one with its own local memory and processor, connected via networking communication. \\
Basically this means that a distributed system is none other than a group of computer communicating via the exchange of messages. \\
Distribution can be at three different levels:
\begin{itemize}
    \item Hardware
    \item Data, which, to be processed, needs to be replicated and partitioned
    \item Control
\end{itemize}
What is the difference between centralized and distributed systems? \\
\begin{table}[h!]
    \centering
    \begin{tabular}{| c | c |}
        \hline
        \multicolumn{2}{| c |}{Differences}\\ 
        \hline
        Distributed & Centralized\\
        \hline
        Autonomous components & Non autonomous components\\
        \hline
        Heterogeneous technology & Most of the time build using homogeneous technology\\
        \hline
        Components can be used exclusively & multiple users share the same resources all times\\
        \hline
        Executed in concurrent processes & single point of control and of failure\\
        \hline
        Multiple points of failure & ;D\\
        \hline
    \end{tabular}
\end{table}
An important question needs to find an answer, why do we bother with distributed systems? Is the added complexity really beneficial? \\
The answer is, obviously, yes (to some extent). 
\begin{itemize}
    \item Distributed systems allow for functional separation between the agents in action.
    \item The model goes toe to toe with the conception of information we have today.
    \item We can distribute computational load over many devices instead of having only one node to complete the task. This is really beneficial for complex tasks.
    \item Distributed systems are very reliable
    \item Sharing a pool of resources with many individuals make them very economical because we don't need to buy many for each one, we just need many resources for many people.
\end{itemize}