\section{Virtualization}
Virtualization is a concept which is closely related to the one of Processes and multithreading, virtualization deals with extending or replacing an existing interface mimicing the behaviour of another system. \\
Virtualization is, nowadays, a standard; that is because it allows to handle hardware changes with ease because the hardware is virtualized and the interface does not change for the above system. With virtualization also come ease of portability and code migration and isolation of failing or attacked components. \\
There are different ways to do virtualization, a couple of interfaces to be virtualized are the following:
\begin{itemize}
    \item Instruction set architecture
    \item System calls
    \item Library calls
    \item Process virtualization in which we have a separate set of instructions, an interpreter/emulator, running atop an OS
    \item Native virtual machine monitor in which we have low level instructions along with barebones minimal operating system
    \item Hosted virtual machine monitor in which we have low level instructions but most of the work is delegated to a full fladged OS.
\end{itemize}
    